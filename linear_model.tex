\documentclass[xetex,mathserif,serif]{beamer}
\usepackage{xeCJK}

\usepackage{listings}
\usepackage{hyperref}
\usepackage{algorithmic,algorithm}
\usepackage{graphicx}
\usepackage{subfigure}
\usepackage{fontspec}
\usepackage{setspace}

\usetheme{Copenhagen}
\usecolortheme{seahorse}
\useoutertheme{infolines}
\useinnertheme{rectangles}
\setmainfont{Times New Roman}

\begin{document}
\title[机器学习读书会报告] % (optional, only for long titles)
{读书报告}
\subtitle{第三章 \ 线性模型}
\author{刘精昌 } % (optional, for multiple authors)
\date{\today}
\subject{读书会}

\begin{frame}
\titlepage
\end{frame}

\section{线性回归模型}

\begin{frame}{引例}
  \begin{block}{工资与教育水平关系}
    考查工人工资水平与其受教育关系:
    \begin{itemize}
      \item[a] 工资水平(每小时美元数):用Y表示
      \item[b] 受教育程度(受教育年数):用X表示
      \item[c] 非可观测因素,如工作经验、天生素质、工作时间等其他因素
    \end{itemize}
  \end{block}
\end{frame}

\begin{frame}{引例}
    \begin{block}{工资与教育水平关系}
    \begin{itemize}
      \item[d] 观测的数据:$\left(x_i,y_i\right),i=1,\cdots,n$ (受访人数),
    \end{itemize}
    \end{block}
    \begin{tabular}{c||c c|c||c c}
      \hline
      % after \\: \hline or \cline{col1-col2} \cline{col3-col4} ...
      $i$ & $x_i$ & $y_i$ & $i$ & $x_i$ & $y_i$ \\
      \hline
      1 & 5.3 & 1.4 & 9 & 8.5 & 3.2 \\
      2 & 11.0 & 3.9 & 10 & 7.1 & 8.6 \\
      3 & 9 & 6.3 & 11 & 15 & 4 \\
      4 & 8.7 & 8.6 & 12 & 12.0 & 9.0 \\
      5 & 10 & 12 & 13 & 29 & 12 \\
      6 & 15.5 & 12 & 14 & 19.7 & 13.1 \\
      7 & 21 & 16 & 15 & 15.1 & 10 \\
      8 & 19 & 14.4 & 16 & 15.7 & 16 \\
      \hline
    \end{tabular}
\end{frame}

\begin{frame}{定义}
\[ y_i = \beta_0+x_{i1}\beta_1+ \dots + x_{i,p-1}\beta_{p-1}+e_i, \ i=1,2,\dots,n\]

\[\left( {\begin{array}{*{20}{c}}
{{y_1}}\\
{{y_2}}\\
 \vdots \\
{{y_n}}
\end{array}} \right) = \left( {\begin{array}{*{20}{c}}
1&{{x_{11}}}& \cdots &{{x_{1,p - 1}}}\\
1&{{x_{11}}}& \cdots &{{x_{2,p - 1}}}\\
 \vdots & \vdots &{}& \vdots \\
1&{{x_{11}}}& \cdots &{{x_{n,p - 1}}}
\end{array}} \right)\left( {\begin{array}{*{20}{c}}
{{\omega _0}}\\
{{\omega _1}}\\
 \vdots \\
{{\omega _{p - 1}}}
\end{array}} \right) + \left( {\begin{array}{*{20}{c}}
{{e_1}}\\
{{e_2}}\\
 \vdots \\
{{e_{p - 1}}}
\end{array}} \right)\]
\end{frame}

\[\bf{y}=\bf{X\beta}+\bf{e}\]

\end{document} 